\documentclass{beamer}
\usepackage{lmodern}
\usepackage{algpseudocode}
\usepackage{algorithm}
\usepackage{amsmath}
\usepackage{amssymb}
\usepackage{mathabx}
\usepackage{graphicx}
\usepackage{chemarrow}
\usepackage[english]{babel}
\usepackage{caption}
\usepackage{epstopdf}
\usepackage{pgfplots}
\usepackage{rotating}
\usepackage{tcolorbox}
\usepackage{slashbox}
\usepackage{tablefootnote}
\usepackage{threeparttable}
\usepackage{array}
\usepackage{siunitx}
\usepackage{lipsum}
\usepackage{adjustbox}
\epstopdfsetup{update}
\usepackage{tikz}
\usetikzlibrary{shapes, arrows, calc}
\usepackage[T1]{fontenc}
\usepackage{lmodern}
\usetikzlibrary{decorations.pathmorphing, decorations.text}
\setbeamertemplate{sidebar right}{}
\setbeamertemplate{footline}{%
\hfill\usebeamertemplate***{navigation symbols}
\hspace{1cm}\insertframenumber{}/\inserttotalframenumber}
\usetikzlibrary{trees}
\DeclareMathOperator*{\argmax}{argmax}
\renewcommand*{\thefootnote}{\fnsymbol{footnote}}
\newcommand{\bigO}[1]{\ensuremath{\mathop{}\mathopen{}\mathcal{O}\mathopen{}\left(#1\right)}}
\newcolumntype{C}[1]{>{\centering\let\newline\\\arraybackslash\hspace{0pt}}m{#1}}
\newcommand{\suchthat}{\, \mid \,} 
\setbeamertemplate{frametitle}{\color{black}\bfseries\insertframetitle\par\vskip-6pt\hrulefill}
\setbeamercolor*{item}{fg=black}

\begin{document}
\begin{frame}[plain]
    \centering{\Huge A Comparison of Antenna Placement Algorithms}
    \vspace{30px}
    \begin{center}
        \Large Abhinav Jauhri, Jason D. Lohn, Derek S. Linden
    \end{center}
\end{frame}

\begin{frame}[t]{Antenna Placement Example}
\begin{figure}
    \centering
    \includegraphics[width=75mm]{car.png}
\end{figure}
\begin{itemize}
    \item \textcolor{red}{$A_1$} has $24$ possible antenna placements
    \item \textcolor{cyan}{$A_2$} has $33$ possible antenna placements
    \item \textcolor{black}{$A_3$} has $136$ possible antenna placements
\end{itemize}
Size of search space $= m^n$\\
\small Goal: Find a placement for each of three antennas.
\end{frame}

\begin{frame}[t]{The Problem}
    Given:
\begin{itemize} \itemsep1.5em
        \item platform $P$ with its surface gridded so that grid points represent possible antenna placement
        \item set of  $m (m > 1)$ antennas $A = {A_1, A_2, \dots, A_m}$
        \item for each $A_i$, a set of $n$ possible placement locations $(n>1)$; $A_i = \{(x_{i1}, y_{i1}, z_{i1}), (x_{i2}, y_{i2}, z_{i2}) \dots (x_{in}, y_{in}, z_{in})\}$
    \end{itemize}
\vspace{10px}
    Find: A set of $m$ optimal antenna locations on $P$
\end{frame}

\begin{frame}[t]{Antenna Placement Objectives}
    \begin{columns}[T]
    \begin{column}{.48\textwidth}
\color{black}\rule{\linewidth}{2pt}
Antenna Placement Issues
\vspace{10px}
\begin{itemize}
    \item[\chemarrow] Coupling among antennas 
    \item[\chemarrow] Parasitic effects and reflections from the host platform 
    \item[\chemarrow] Loss of efficiency
    \item[\chemarrow] Difficulty conforming to aerodynamic, thermal, other enovironment factors 
\end{itemize} 
\end{column}%
\hfill%
\begin{column}{.48\textwidth}
\color{black}\rule{\linewidth}{2pt}
Desired Antenna Placement Objectives
\vspace{10px}
\begin{itemize}
    \item[\chemarrow] Gain in radiation pattern 
    \item [\chemarrow] Minimize coupling
\end{itemize}
\end{column}%
\end{columns}
\end{frame}

\begin{frame}{Fitness Functions - I}
    To minimize difference in gain:
\begin{tcolorbox}[colback=green!5]
\begin{equation} \label{eq:rp}
  F_{RP}(A_i) = \sum_{\theta}\sum_{\phi} 
           \| ISG_i(\theta,\phi) - FSG_i(\theta,\phi) \| ^2,
\end{equation}
\end{tcolorbox}
where
\begin{itemize}
        \small
    \item $\theta, \phi$ spherical and cylindrical coordinates
    \item $ISG(\cdot)$ returns in-situ gain pattern
    \item $FSG(\cdot)$ returns free-space gain pattern  
\end{itemize}
\begin{figure}
    \centering
    \includegraphics[width=40mm, height=30mm]{rad.png}
\end{figure}

\end{frame}

\begin{frame}{Fitness Functions - II}
    To minimize coupling: 
\begin{tcolorbox}[colback=green!5]
\begin{equation}
  F_{MC} = \sum_{i=1}^{m-1}\sum_{j=i+1}^{m} CP(A_i, A_j),
\end{equation}
\end{tcolorbox}
where
\begin{itemize}
    \item $CP(\cdot)$ computes the coupling between two antennas
\end{itemize}
\end{frame}

\begin{frame}{Overall Fitness Function}
For an individual/hypothesis, fitness is defined as:
\begin{tcolorbox}[colback=green!5]
\begin{equation} \label{eq:optimal}
  F = \alpha F_{MC} + \beta \sum_{i} F_{RP}(A_i),
\end{equation}
\end{tcolorbox}
where $\alpha + \beta = 1$
\end{frame}

\begin{frame}[t]{Experimental Setup}
    \begin{itemize}
        \item Create individual(s) such that each individual is defined by a placement for each of the $m$ antennas
        \item Run all individuals through \textit{NEC} simulator \footnote{http://www.nec2.org} to get fitness parameters 
        \item Apply EA operators 
        \item Repeat\dots 
    \end{itemize}
    \vspace{10mm}
    Algorithms explored: Simple GA, Simulated Annealing, Evolutionary Strategy, and Hill Climber \footnote{https://github.com/ajauhri/evol-ant-placement} 
\end{frame}
\begin{frame}{Experiments: Test Cases}
\begin{table}
\centering
\begin{tabular}{|c|c|c|} \hline
    ID&Antennas&Total allowable placements\\ \hline
tc1 & 2 & 7,056 (83x83) \\ \hline
tc2 & 3 & 50,625 (45x45x25) \\ \hline
tc3 & 3 & 126,025 (71x71x25) \\ \hline
tc4 & 4 & 20,736 (12x12x12x12) \\
\hline\end{tabular}
\end{table}
\small *Allowable placements for each antenna are provided within parenthesis
\end{frame}

\begin{frame}{Results: Mean Evaluations}
    Mean number of evaluations to reach the best solution (over 10 runs):
\begin{table}
\centering
\begin{tabular}{|>{\small}c|>{\small}c|>{\small}c|>{\small}c|c|} \hline
\centering
\backslashbox{test case}{method} & GA & ES & SA & HC\\\hline
tc1(7056)\footnote{Total number of possible evaluations within parenthesis} & \num{2350} & \num{1728} & \num{667} & \num{164} \\ \hline
tc2(50,625) & \num{31680} & \num{11165} & \num{1653} & \num{174} \\ \hline
tc3(126,025) & \num{45900} & \num{26880} & \num{4809} & \num{227} \\ \hline
tc4(20,736) & \num{6150} & \num{4466} & \num{423} & \num{90} \\ \hline
\end{tabular}
\end{table}
\begin{itemize}
    \item Simulated Annealing was the fastest
    \item Evolutionary Strategy always found optimal, but was relatively slow
\end{itemize}
\end{frame}

\begin{frame}{Results - Mean Best Fitness}
\small
\begin{itemize}
    \item[--] Lower fitness is better
    \item[--] Mean taken over $10$ runs
    \item[--] In all test cases, antennas were subjected to same frequency
\end{itemize}

\vspace{10px}
    \resizebox{\linewidth}{!}{
\begin{tabular}{|>{\small}c|>{\small}c|>{\small}c|>{\small}c|>{\small}c|c|} \hline
\backslashbox{test case}{method} & Exhaustive($H^*$) & GA & ES & SA & HC\\\hline
test case 1 & \num{0.4968} & \num{0.4993} & \num{0.4968} & \num{0.4994} & \num{0.5015} \\ \hline
test case 2 & \num{0.4968} & \num{0.4979} & \num{0.4968} & \num{0.5042} & \num{0.5138} \\ \hline
test case 3 & \num{0.4974} & \num{0.4976} & \num{0.4974} & \num{0.4974} & \num{0.4987}\footnote{Best performing individuals lie within a small fitness range} \\ \hline
test case 4 & \num{0.4992} & \num{0.4992} & \num{0.4992} & \num{0.4992} & \num{0.4992} \\ \hline
\end{tabular}}

\end{frame}


\begin{frame}{Equivalence of fitness to efficiency}
\small For a particular test case, fitness change of $0.01$ is equivalent to either the corresponding value under expected gain ($\mathbb E_g$) column, or difference in coupling ($\Delta_c$).
\begin{table}
\centering
  \begin{threeparttable}
      \begin{tabular}{|C{1cm}|C{2.5cm}|C{2.5cm}|} \hline
          ID& $\mathbb E_g$ & $\Delta_{c}$ (dB) \\ \hline
tc1 & 872.277 & 0.5474 \\ \hline
tc2 & 862.082 & 1.3034 \\ \hline
tc3 & 861.845 & 1.5180 \\ \hline
tc4 & 871.049 & 0.5693 \\
\hline\end{tabular}
\end{threeparttable}
\end{table}
\tiny
    $\mathbb E_g = \frac{1}{N \cdot m} \sum_{i}^m F_{RP}(A_i),$
where $N = \;\mid \theta \mid \cdot \mid \phi \mid$
\end{frame}
\begin{frame}
    Thanks!
\end{frame}
\end{document}
